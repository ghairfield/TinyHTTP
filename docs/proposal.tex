\documentclass[12pt,letterpaper]{extarticle}
\usepackage{amsmath}
\usepackage{amsfonts}
\usepackage{amssymb}
\usepackage{graphicx}
\usepackage{color}
\usepackage{mwe} % Graphics placeholder
\usepackage[colorlinks=true,linkcolor=red,citecolor=blue]{hyperref}
\usepackage{float}
\usepackage{listings}
\usepackage{setspace}

\definecolor{mygreen}{rgb}{0,0.6,0}
\definecolor{mygray}{rgb}{0.5,0.5,0.5}

\interfootnotelinepenalty=10000

\lstset{
	basicstyle=\ttfamily,
	keywordstyle=\color{blue}\ttfamily,
	stringstyle=\color{red}\ttfamily,
	commentstyle=\color{mygreen},
	captionpos=b,
	numbers=left,
	numberstyle=\tiny\color{mygray},
	showspaces=false,
	showstringspaces=false,
	showtabs=false,
  columns=fixed
}
\lstMakeShortInline[columns=fixed]| % This allows for |code| 
\graphicspath{{./images/}{IR}}
\usepackage[left=2.0cm, top=2.0cm]{geometry}

\title{CS410P Proposal}
\author{Greg Hairfield}
\date{April 27, 2021}

\doublespacing
\begin{document}

\maketitle

I plan on creating a HTTP/1.0 server for this project. The current HTTP/1.0
\href{https://www.ietf.org/rfc/rfc1945.txt}{specification} is well documented
and available to anyone. Ideally the server will validate a HTTP request and 
formulate a valid response. The response will only serve static resources
(e.g. HTML page, text document). I plan on using two Rust crates:
\href{https://clap.rs/}{Clap} to help with command line parsing and 
\href{https://docs.rs/toml/0.5.8/toml/}{TOML} to parse configuration files.

Features of HTTP/1.0 I plan on implementing:

\begin{center}
  \begin{tabular}{ll}
    \hline
    Feature & Feature \\
    \hline\hline
    HTTP Version & Uniform Resource Identifiers \\
    \hline
    Date/Time Formats & Media Types \\
    \hline
    HTTP Message & Request \\
    \hline
    Response & Method Definitions \\
    \hline
    Status Codes & Header Fields \\
    \hline
  \end{tabular}
\end{center}

As above noted, I do \emph{not} plan on implementing additional features such
as HTTPS, streaming, proxy,
gateway, tunnel or caching for this project.

Project information:

\begin{center}
  \begin{tabular}{rl}
    Name: & Greg Hairfield \\
    Email: & hairfiel@pdx.edu \\
    Project Name: & Tiny HTTP \\
    Git URL: & https://github.com/ghairfield/TinyHTTP \\
  \end{tabular}
\end{center}

\end{document}
